\acp{SoC} are devices that include several circuits with different functionalities cooperating to perform a given computational workload. For instance, Xilinx and Intel produce \acp{SoC} that integrate both hard \acp{CPU} and \acp{FPGA}, communicating through on-chip high-performance interfaces. Such interfaces usually allow for much higher data rates than off-chip connections, such as \ac{PCI}, and also lower latencies. The work proposed in this document aims at assessing the performance of the communication channels between the \acp{CPU} and the \ac{FPGA} fabric of two \ac{SoC} devices produced by Xilinx and Intel. The obtained performance measurements are compared with the theoretical bandwidth of the devices as advertised by the producing brands, and the methodology for reproducing the results is briefly explained.