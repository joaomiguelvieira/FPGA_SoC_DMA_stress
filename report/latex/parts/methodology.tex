\section{Methodology}\label{sec:methodology}

To evaluate the duplex, \ac{PS}-to-\ac{PL}, and \ac{PL}-to-\ac{PS} bandwidths of the on-chip high-performance communication channels of the Xilinx and Intel's \ac{FPGA}-\acp{SoC}, three simple systems were envisioned. These architectures consist of one or more \ac{DMA} engines connected to the high-performance ports (\ac{HP} on the Xilinx device and \ac{F2S} on the Intel device), their respective control interfaces connected to a lightweight port (\ac{GP} on Xilinx devices and \ac{F2H} on Intel devices), and some extra hardware to either loop data back, absorb an incoming stream or produce a data stream on the reconfigurable logic. The top-level architectures of the proposed systems are depicted in Figure~\ref{fig:evaluation_circuit}.

\begin{figure}[t]
    \centering
    \begin{subfigure}[m]{.5\linewidth}
        \centering
        {\footnotesize
\begin{tikzpicture}[scale=.45]
    \draw[thick] (-1,3.5) coordinate (top) -- ++(0,-11);
    \draw[thick] (0,0) coordinate (origin) rectangle (6,1);
    \foreach \i in {1,...,2} {
        \draw (\i,0) -- ++(0,1);
    }
    \draw (5,0) -- ++(0,1);
    \draw (3.5,.5) node{$\cdots$};
    \draw (3,0) coordinate (aux1) edge[<->,>=stealth,thick] ++(0,-1) ++(0,-1) coordinate (aux2);
    \draw[thick] (origin |- aux2) coordinate (aux3) rectangle ++(6,-6) coordinate (aux4);
    \foreach \i in {0,...,3} {
        \draw (aux3 -| aux4) ++(0,-.5-1.5*\i) edge[->,>=stealth,thick] ++(1,0);
        \draw[fill=black] (aux3 -| aux4) ++(1,-1.5*\i) rectangle ++(.25,-1) coordinate (aux5);
        \draw (aux5) rectangle ++(2,1) coordinate (aux6);
        \draw[fill=black] (aux6) rectangle ++(.25,-1) ++(0,.5) coordinate (aux7);
        \draw ($(aux5)!.5!(aux6)$) node{FIFO};
        \path (aux7) edge[thick] ++(.5,0)++(.5,0) edge[thick] ++(0,-.75)++(0,-.75) edge[->,>=stealth,thick] ++(-4,0);
        \draw (aux3) ++(0,-.5-1.5*\i) edge[<-,>=stealth,thick] ++(-1,0);
        \draw (aux3) ++(0,-1.25-1.5*\i) edge[->,>=stealth,thick] ++(-1,0) ++(-1,-.125) coordinate (aux8);
        \draw[fill=black] (aux8) rectangle ++(-.1,1) coordinate (aux9);
        \draw ($(aux8 -| aux9)!.5!(aux9)$) node[anchor=east]{\ifthenelse{\i=2}{$\vdots$}{\ifthenelse{\i=3}{$N$}{$\i$}}};
    }
    \draw[fill=black] (-1,0) coordinate (aux10) rectangle ++(-.1,1) coordinate (aux11);
    \draw ($(aux10 -| aux11)!.5!(aux11)$) node[anchor=east]{\parbox{3cm}{\begin{flushright}Lightweight Interface for Control and Monitoring\end{flushright}}};
    \draw[decoration={brace,mirror,raise=5pt},decorate] (-1.7,-1.5) -- node[left=6pt] {\parbox{2.5cm}{\begin{flushright}High-Performance Interface(s) for Data Transfer\end{flushright}}} (-1.7,-7);
    \draw ($(aux3)!.5!(aux4)$) node{\parbox{2.5cm}{\centering \acf{DMA} Engine}};
    \draw (3,1) node[anchor=south]{Control Registers};
    \draw (top) node[anchor=north east]{\textbf{Processing System}};
    \draw (top) node[anchor=north west]{\textbf{Programmable Logic}};
    \draw (-1,2.5) edge[->,thick,>=stealth] ++(8,0);
    \draw (-1,2.5) edge[->,thick,>=stealth] ++(-7.3,0);
    \draw ($(aux11 -| aux10)!.5!(aux10)$) edge[<->,thick,>=stealth] ++(1,0);
\end{tikzpicture}
}
        \caption{}
        \label{fig:evaluation_duplex}
    \end{subfigure}\hfill
    \begin{subfigure}[m]{.5\linewidth}
        \centering
        {\footnotesize
\begin{tikzpicture}[scale=.45]
    \draw[thick] (-1,3.5) coordinate (top) -- ++(0,-11);
    \draw[thick] (0,0) coordinate (origin) rectangle (6,1);
    \foreach \i in {1,...,2} {
        \draw (\i,0) -- ++(0,1);
    }
    \draw (5,0) -- ++(0,1);
    \draw (3.5,.5) node{$\cdots$};
    \draw (3,0) coordinate (aux1) edge[<->,>=stealth,thick] ++(0,-1) ++(0,-1) coordinate (aux2);
    \draw[thick] (origin |- aux2) coordinate (aux3) rectangle ++(6,-6) coordinate (aux4);
    \foreach \i in {0,...,3} {
        \draw (aux3 -| aux4) ++(0,-.9-1.5*\i) edge[->,>=stealth,thick] ++(1,0) ++(1,0) node[anchor=west]{\textit{\{void\}}};
        \draw[fill=black] (aux3 -| aux4) ++(1,-1.5*\i) ++(.25,-1) coordinate (aux5);
        \draw (aux5) ++(2,1) coordinate (aux6);
        \draw[fill=black] (aux6) ++(.25,-1) ++(0,.5) coordinate (aux7);
        \draw (aux3) ++(0,-.9-1.5*\i) edge[<-,>=stealth,thick] ++(-1,0);
        \draw (aux3) ++(0,-1.25-1.5*\i) ++(-1,-.125) coordinate (aux8);
        \draw[fill=black] (aux8) rectangle ++(-.1,1) coordinate (aux9);
        \draw ($(aux8 -| aux9)!.5!(aux9)$) node[anchor=east]{\ifthenelse{\i=2}{$\vdots$}{\ifthenelse{\i=3}{$N$}{$\i$}}};
    }
    \draw[fill=black] (-1,0) coordinate (aux10) rectangle ++(-.1,1) coordinate (aux11);
    \draw ($(aux10 -| aux11)!.5!(aux11)$) node[anchor=east]{\parbox{3cm}{\begin{flushright}Lightweight Interface for Control and Monitoring\end{flushright}}};
    \draw[decoration={brace,mirror,raise=5pt},decorate] (-1.7,-1.5) -- node[left=6pt] {\parbox{2.5cm}{\begin{flushright}High-Performance Interface(s) for Data Transfer\end{flushright}}} (-1.7,-7);
    \draw ($(aux3)!.5!(aux4)$) node{\parbox{2.5cm}{\centering \acf{DMA} Engine}};
    \draw (3,1) node[anchor=south]{Control Registers};
    \draw (top) node[anchor=north east]{\textbf{Processing System}};
    \draw (top) node[anchor=north west]{\textbf{Programmable Logic}};
    \draw (-1,2.5) edge[->,thick,>=stealth] ++(8,0);
    \draw (-1,2.5) edge[->,thick,>=stealth] ++(-7.3,0);
    \draw ($(aux11 -| aux10)!.5!(aux10)$) edge[<->,thick,>=stealth] ++(1,0);
\end{tikzpicture}
}
        \caption{}
        \label{fig:evaluation_ps2pl}
    \end{subfigure}\\\medskip
    \begin{subfigure}[m]{.5\linewidth}
        \centering
        {\footnotesize
\begin{tikzpicture}[scale=.45]
    \draw[thick] (-1,3.5) coordinate (top) -- ++(0,-11);
    \draw[thick] (0,0) coordinate (origin) rectangle (6,1);
    \foreach \i in {1,...,2} {
        \draw (\i,0) -- ++(0,1);
    }
    \draw (5,0) -- ++(0,1);
    \draw (3.5,.5) node{$\cdots$};
    \draw (3,0) coordinate (aux1) edge[<->,>=stealth,thick] ++(0,-1) ++(0,-1) coordinate (aux2);
    \draw[thick] (origin |- aux2) coordinate (aux3) rectangle ++(6,-6) coordinate (aux4);
    \foreach \i in {0,...,3} {
        \draw (aux3 -| aux4) ++(0,-.9-1.5*\i) edge[<-,>=stealth,thick] ++(1,0) ++(1,-.5) coordinate (corner1) rectangle ++(6,1) coordinate (corner2);
        \draw ($(corner1)!.5!(corner2)$) node{Stream Generator};
        \draw[fill=black] (aux3 -| aux4) ++(1,-1.5*\i) ++(.25,-1) coordinate (aux5);
        \draw (aux5) ++(2,1) coordinate (aux6);
        \draw[fill=black] (aux6) ++(.25,-1) ++(0,.5) coordinate (aux7);
        \draw (aux3) ++(0,-.9-1.5*\i) edge[->,>=stealth,thick] ++(-1,0);
        \draw (aux3) ++(0,-1.25-1.5*\i) ++(-1,-.125) coordinate (aux8);
        \draw[fill=black] (aux8) rectangle ++(-.1,1) coordinate (aux9);
        \draw ($(aux8 -| aux9)!.5!(aux9)$) node[anchor=east]{\ifthenelse{\i=2}{$\vdots$}{\ifthenelse{\i=3}{$N$}{$\i$}}};
    }
    \draw[fill=black] (-1,0) coordinate (aux10) rectangle ++(-.1,1) coordinate (aux11);
    \draw ($(aux10 -| aux11)!.5!(aux11)$) node[anchor=east]{\parbox{3cm}{\begin{flushright}Lightweight Interface for Control and Monitoring\end{flushright}}};
    \draw[decoration={brace,mirror,raise=5pt},decorate] (-1.7,-1.5) -- node[left=6pt] {\parbox{2.5cm}{\begin{flushright}High-Performance Interface(s) for Data Transfer\end{flushright}}} (-1.7,-7);
    \draw ($(aux3)!.5!(aux4)$) node{\parbox{2.5cm}{\centering \acf{DMA} Engine}};
    \draw (3,1) node[anchor=south]{Control Registers};
    \draw (top) node[anchor=north east]{\textbf{Processing System}};
    \draw (top) node[anchor=north west]{\textbf{Programmable Logic}};
    \draw (-1,2.5) edge[->,thick,>=stealth] ++(8,0);
    \draw (-1,2.5) edge[->,thick,>=stealth] ++(-7.3,0);
    \draw ($(aux11 -| aux10)!.5!(aux10)$) edge[<->,thick,>=stealth] ++(1,0);
\end{tikzpicture}
}
        \caption{}
        \label{fig:evaluation_pl2ps}
    \end{subfigure}
    \caption{Systems proposed to measure the (\subref{fig:evaluation_duplex}) duplex, (\subref{fig:evaluation_ps2pl}) \ac{PS}-to-\ac{PL}, and (\subref{fig:evaluation_pl2ps}) \ac{PL}-to-\ac{PS} bandwidths of the \ac{FPGA}-\ac{SoC} devices' on-chip high-performance interfaces.}
    \label{fig:evaluation_circuit}
\end{figure}

For comparison purposes, this work uses devices of the same family than those used in~\cite{DBLP:conf/arc/GobelECMJ17}. More specifically, the Xilinx ZYNQ-7010 (Zybo board from Digilent~\cite{zybo}) and the Intel Cyclone V SE (DE1-SoC from TerasIC~\cite{de1soc}). The testing systems were developed using \ac{VHDL} language and were targeted in both the Xilinx and the Intel devices with the same operating frequency to ensure a fair comparison.

In the software-side, simple programs will be written in C language and targeted in both devices to control the hardware structures in the reconfigurable logic and output the results of the experiment.

In summary, the main objectives of this project are:
\begin{enumerate}[nosep]
    \item Learning to use Xilinx and Intel's platforms for \ac{FPGA} and \ac{FPGA}-\ac{SoC} development (Xilinx Vivado and Intel Quartus Prime, respectively);
    \item Researching architectures of \ac{DMA} devices capable of stressing the on-chip high-performance communication interfaces of the \ac{FPGA}-\ac{SoC} devices;
    \item Producing a system capable of evaluating the performance of the on-chip high-performance communication interfaces of the \ac{FPGA}-\ac{SoC} devices, as well as supporting software;
    \item Comparing the obtained results with the theoretical limits of the devices and the results of previous studies.
\end{enumerate}
