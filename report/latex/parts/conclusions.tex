\section{Conclusions}\label{sec:conclusions}

This work aimed at evaluating the performance of the on-chip high-performance interfaces of Xilinx and Intel's low-end \ac{FPGA}-\ac{SoC} device families (ZYNQ-7000 and Cyclone V, respectively). For achieving that goal, three systems were designed to fully exploit the duplex, \ac{PS}-to-\ac{PL}, and \ac{PL}-to-\ac{PS} bandwidths. Then, those systems were implemented and targeted on both Xilinx ZYNQ-7010 and Intel Cyclone V SE devices. Multiple implementation options were considered regarding the choice of \ac{DMA} engines to use on each device and the software environment running on the hard \acp{CPU} of the \ac{FPGA}-\acp{SoC} (bare-metal or Linux).

The obtained results were compared to the results obtained by G{\"{o}}bel \textit{et al.} in a previous study, and also with the theoretical limits of the devices. The maximum observed data rates are similar to the ones observed by G{\"{o}}bel \textit{et al.} Additionally, the results suggest that the bandwidth allowed by the on-chip high-performance channels of both devices is limited by the memory controller. Whenever getting close to the DDR maximum data rate, the bandwidth starts to degrade. Furthermore, the higher bandwidths achieved by single-sided transfers and the higher standard deviations associated with experiments using sub-optimal configurations suggest that there is some level of entropy at the memory controller level that affects the performance negatively.

The artifact produced by this work and its support documentation can be found at \url{https://github.com/joaomiguelvieira/FPGA_SoC_DMA_stress/}.