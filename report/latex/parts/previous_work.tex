\section{Previous Work}\label{sec:previous_work}

Several efforts have been made towards analyzing the performance of \ac{FPGA}-\acp{SoC}' memory hierarchies and on-chip communication channels. The first known results were presented by Sadri \textit{et al.}~\cite{sadri2013energy}. They assessed the performance of the \ac{ACP} in the Xilinx ZYNQ-7020 device, which, theoretically, is capable of the same data rate than a single \ac{HP} port. Their results show that the \ac{ACP} port achieves a duplex throughput of \SI{1.7}{\giga\byte\per\second} when connected to a \ac{FPGA}-implemented device operating at \SI{125}{\mega\hertz}.

Sklyarov \textit{et al.}~\cite{DBLP:conf/dsd/SklyarovSSS15} also focused their efforts in evaluating the high-performance interfaces between the \ac{PS} and the \ac{PL} on the Xilinx ZYNQ-7000 device family. Although they do not explicitly show the maximum bandwidth attained at \SI{100}{\mega\hertz}, it can be derived from the results. Using a single 64-bit \ac{HP} port at \SI{100}{\mega\hertz}, their system achieved a maximum throughput of \SI{284}{\mega\byte\per\second}, which is significantly lower than the theoretically possible \SI{800}{\mega\byte\per\second}.

Also focusing the Xilinx ZYNQ-7000 device family, Tahghighi \textit{et al.}~\cite{tahghighi2016analytical} developed a mathematical model that allows estimating the latency of the memory accesses from the \ac{PL}. Although their model considers several important parameters, it does not include the combination of several \ac{HP} ports to increase the bandwidth, which limits its scope. Furthermore, it is limited to the Xilinx ZYNQ-7000 device family, not allowing to draw conclusions for different devices.

In addition to \cite{sadri2013energy,DBLP:conf/dsd/SklyarovSSS15,tahghighi2016analytical}, there are other works aiming at assessing the performance of the on-chip communication channels of the Xilinx ZYNQ-7000 devices~\cite{svensson2016exploring}. However, there has not been an equivalent effort to evaluate these interfaces on Intel \ac{FPGA}-\acp{SoC}. G{\"{o}}bel \textit{et al.}~\cite{DBLP:conf/arc/GobelECMJ17} presented one of the few studies that also evaluate the performance of the on-chip communication channels on Intel \ac{FPGA}-\acp{SoC}. Their work focus on accessing the performance of the on-chip high-performance channels between the hard \acp{CPU} and the reconfigurable logic on both Xilinx and Intel low-end device families (ZYNQ-7000 and Cyclone V, respectively). Their assessment is divided into two phases. First, they stress the on-chip high-performance communication channels of the devices and register the maximum achieved data rates. Second, they assess the channels' performance in the context of video processing applications. The expected \ac{PS}-to-\ac{PL} (\ac{HPS}-to-\ac{FPGA} in Intel's notation) bandwidth of the high-performance channels and the results obtained in their experiments are summarized in Table~\ref{tab:results_gobel}.

\begin{table}[b]
    \centering
    \caption{Results obtained by G{\"{o}}bel \textit{et al.}~\cite{DBLP:conf/arc/GobelECMJ17} regarding the bandwidth of the high-performance on-chip communication channels of two Xilinx and Intel's \ac{FPGA}-\acp{SoC} (ZYNQ-7020 and Cyclone V SE, respectively).}
    \label{tab:results_gobel}
    \begin{tabular}{cl|r|r|}
    \cline{3-4}
                                                                                                                                                 & \multicolumn{1}{c|}{}                                   & \multicolumn{1}{c|}{\textbf{Xilinx ZYNQ-7000}} & \multicolumn{1}{c|}{\textbf{Intel Cyclone V}} \\ \hline
    \multicolumn{1}{|c|}{\multirow{3}{*}{\textbf{DRAM}}}                                                                                         & \textbf{Bandwidth [\si{\mega\transfer\per\second}]}     & \textbf{1066}                                  & \textbf{800}                                  \\ \cline{2-4} 
    \multicolumn{1}{|c|}{}                                                                                                                       & \textbf{Transfer width [\si{bit}]}                      & 32                                             & 32                                            \\ \cline{2-4} 
    \multicolumn{1}{|c|}{}                                                                                                                       & \textbf{Bandwidth [\si{\mega\byte\per\second}]}         & \textbf{4264}                                  & \textbf{3200}                                 \\ \hline
    \multicolumn{2}{|c|}{\textbf{AXI3 Interface Frequency [\si{\mega\hertz}]}}                                                                                                                             & 110                                            & 110                                           \\ \hline
    \multicolumn{1}{|c|}{\multirow{3}{*}{\textbf{\begin{tabular}[c]{@{}c@{}}Theoretical\\ Bandwidth [\si{\mega\byte\per\second}]\end{tabular}}}} & \textbf{1 \texttimes{} \ac{HP} @ \SI{110}{\mega\hertz}} & 880                                            & 880                                           \\ \cline{2-4} 
    \multicolumn{1}{|c|}{}                                                                                                                       & \textbf{2 \texttimes{} \ac{HP} @ \SI{110}{\mega\hertz}} & 1760                                           & 1760                                          \\ \cline{2-4} 
    \multicolumn{1}{|c|}{}                                                                                                                       & \textbf{4 \texttimes{} \ac{HP} @ \SI{110}{\mega\hertz}} & \textbf{3520}                                  & \textbf{3200}                                 \\ \hline
    \multicolumn{1}{|c|}{\multirow{3}{*}{\textbf{\begin{tabular}[c]{@{}c@{}}Real\\ Bandwidth [\si{\mega\byte\per\second}]\end{tabular}}}}        & \textbf{1 \texttimes{} \ac{HP} @ \SI{110}{\mega\hertz}} & 879                                            & 879                                           \\ \cline{2-4} 
    \multicolumn{1}{|c|}{}                                                                                                                       & \textbf{2 \texttimes{} \ac{HP} @ \SI{110}{\mega\hertz}} & 1723                                           & 1723                                          \\ \cline{2-4} 
    \multicolumn{1}{|c|}{}                                                                                                                       & \textbf{4 \texttimes{} \ac{HP} @ \SI{110}{\mega\hertz}} & \textbf{3499}                                  & \textbf{2715}                                 \\ \hline
    \end{tabular}
\end{table}

In their experiments, G{\"{o}}bel \textit{et al.} were able to almost achieve the theoretical maximum bandwidth provided by the four \ac{HP} ports of the Xilinx device for the testing frequency (\SI{110}{\mega\hertz}). However, they could not achieve the theoretical maximum data rate of the 256-bit \ac{F2S} port in the Intel Cyclone V device. They explain that this effect is probably related to the block size used for the assessment. It is also noticeable that while the maximum throughput of the \ac{HP} ports in the ZYNQ-7000 device is bounded by the AXI3 interface operation frequency, the \ac{F2S} interface of the Cyclone V device is bounded by the SDRAM maximum bandwidth. This is caused by the lower data rate of the SDRAM connected to the Cyclone V device (\SI{800}{\mega\transfer\per\second}) compared to the one connected to the ZYNQ-7000 device (\SI{1066}{\mega\transfer\per\second}).

As the work of G{\"{o}}bel \textit{et al.} is the most comprehensive study on the performance of the on-chip high-performance interfaces that considers both Xilinx and Intel \ac{FPGA}-\acp{SoC} at the date this document is written, it will be used as a reference.

To simplify the notation used in the rest of this document, whenever both Xilinx and Intel devices are referred, only the Xilinx terms will be explicitly written. For example, ``\ac{PS}-to-\ac{PL} (\ac{HPS}-to-\ac{FPGA} in Intel's notation) bandwidth'' will be simply referred to as ``\ac{PS}-to-\ac{PL} bandwidth''.

The next Section briefly explains the evaluation methodology used in this work.